\documentclass[11pt,a4paper]{article}

\usepackage{listings}
\usepackage{color}
\usepackage{code}
\usepackage{graphicx}
\usepackage{hyperref}

\usepackage{natbib}
\bibpunct();A{},
\let\cite=\citep

%include lhs2TeX.fmt

% -----------------------------------------------------------------------------=
% Haskell listing style

% Taken from
%  http://ulissesaraujo.wordpress.com/2008/03/30/latex-listings-package-haskell/
% and modified (quite a lot)

\definecolor{lightgray}{gray}{0.96}
\definecolor{gray_ulisses}{gray}{0.55}
\definecolor{castanho_ulisses}{rgb}{0.71,0.33,0.14}
\definecolor{preto_ulisses}{rgb}{0.41,0.20,0.04}
\definecolor{green_ulises}{rgb}{0.2,0.75,0}

\lstdefinelanguage{HaskellUlisses} {
        basicstyle=\ttfamily\footnotesize,
	sensitive=true,
        morecomment=[l][\color{gray_ulisses}\ttfamily]{--},
        morecomment=[s][\color{gray_ulisses}\ttfamily]{\{-}{-\}},
	morestring=[b]",
	stringstyle=\color{red},
	showstringspaces=false,
        numberblanklines=false,
	showspaces=false,
	breaklines=true,
	showtabs=false,
        backgroundcolor=\color{lightgray},
        emph=
        {[1]
		case,class,data,deriving,do,else,if,import,in,infixl,infixr,instance,let,
                module,of,primitive,then,type,where,foreign,import,export,ccall
        },
        emphstyle={[1]\color{blue}},
	emph=
	{[2]
        },
	emphstyle={[2]\color{castanho_ulisses}},
	emph=
	{[3]
        },
	emphstyle={[3]\color{preto_ulisses}\textbf},
	emph=
	{[4]
        },
	emphstyle={[4]\color{castanho_ulisses}\textbf},
	emph=
	{[5]
        },
	emphstyle={[5]\color{preto_ulisses}\textbf}
}

\lstdefinestyle{numbers}
  {numbers=left, stepnumber=1, numberstyle=\color{gray_ulisses}\tiny, numbersep=10pt}
\lstdefinestyle{nonumbers}
  {numbers=none}

\lstnewenvironment{haskell}
{\lstset{language=HaskellUlisses,style=nonumbers}}
{}

\lstnewenvironment{numhaskell}
{\lstset{language=HaskellUlisses,style=numbers}}
{}

\lstnewenvironment{floathaskell}
{\lstset{language=HaskellUlisses,style=numbers}}
{}



\newcommand{\comment}[1]{}

\newcommand{\Section}[2]{\section{#2}\label{sec:#1}}
\newcommand{\Subsection}[2]{\subsection{#2}\label{sec:#1}}
\newcommand{\Subsubsection}[2]{\subsubsection{#2}\label{sec:#1}}
\newcommand{\secref}[1]{Section~\ref{sec:#1}}
\newcommand{\figref}[1]{Figure~\ref{fig:#1}}
\newcommand{\lstref}[1]{Listing~\ref{lst:#1}}

\title{Parallel and Concurrent Programming in Haskell
  \\ \normalsize{Lab Exercises}}

\author{Simon Marlow\\\texttt{simonmar@microsoft.com}\\Microsoft Research Ltd., Cambridge, U.K.}

\begin{document}

\maketitle
\makeatactive

% \begin{abstract}
% \end{abstract}

\Section{intro}{Introduction}

Lines beginning with @$@ are commands that you type at the
command-line (a terminal in Linux, or a command-prompt in Windows).
For example:

{\small \begin{verbatim}
  $ ghc-pkg list
\end{verbatim}}

\noindent to see what Haskell packages are installed.

If that doesn't work, you may need to add the relevant directories to
 your @PATH@:

{\small \begin{verbatim}
$ PATH=/export/opt/HASKELL/2011.4.0.0/bin:/export/opt/GHC/7.4.1/bin:$PATH
\end{verbatim}}

Next, install the extra packages that we'll need for the exercises
using @cabal@:

\begin{verbatim}
$ cabal update
$ cabal install monad-par-0.1.0.3
$ cabal install remote-0.1.1 derive-2.5.8
$ cabal install parallel-3.2.0.2
$ cabal install HTTP-4000.2.3 xml-1.3.12
\end{verbatim}

Next, get the sample code and unpack it.

{\small \begin{verbatim}
wget http://community.haskell.org/~simonmar/par-tutorial-cadarache.tar.gz
tar xvzf par-tutorial-cadarache.tar.gz
\end{verbatim}}

\noindent Build the code.  There is a @Makefile@, so you should be
able to say @make@ to build all the programs:

{\small \begin{verbatim}
  $ cd par-tutorial/code
  $ make
\end{verbatim}}

\noindent Alternatively, you can compile any individual program like this:

{\small \begin{verbatim}
  $ ghc -threaded -rtsopts -eventlog -O2 sudoku1.hs
\end{verbatim}}

\newpage\Section{par}{Lab 1: Parallel Haskell}

Test that you can get parallel speedup.  Compile @sudoku1@ as above,
and run it on one processor like so:

{\small \begin{verbatim}
  $ ./sudoku1 sudoku17.1000.txt +RTS -s
\end{verbatim}}

\noindent Next, the static parallel version:

{\small \begin{verbatim}
  $ ./sudoku2 sudoku17.1000.txt +RTS -s -N2
\end{verbatim}}

\noindent Next, the @parMap@ version:

{\small \begin{verbatim}
  $ ./sudoku3 sudoku17.1000.txt +RTS -s -N2
\end{verbatim}}

\noindent (the last one should be the fastest)

\paragraph{Exercise 1.1.} This exercise is to speed up a program that
performs full-text indexing and searching of documents.  The
sequential program can be found in @code/index/index.hs@, and you will
need to download and unpack the set of sample documents:

\begin{verbatim}
$ wget http://community.haskell.org/~simonmar/par-tutorial-ex1.1-docs.tar.bz2
$ tar xvjf par-tutorial-ex1.1-docs.tar.bz2
\end{verbatim}

The sample documents are a selection of messages from the
@haskell-cafe@ mailing list.  Try out the program like this:

\begin{verbatim}
$ ghc -O index.hs
$ ./index docs/*
search (^D to end): <type your search terms here>
\end{verbatim}

\noindent For example, you could enter "parallel concurrent" and the
program would list the filenames of all the documents that contain
both those words.  To benchmark the program, run it like this:

\begin{verbatim}
$ echo "concurrent parallel" | ./index docs/* +RTS -s
\end{verbatim}

The program works by creating a mapping from words to the set of
documents that contains that word.  Documents are numbered by their
order on the command-line, so that we can represent a set of documents
by @IntSet@.

\begin{haskell}
type DocSet = IntSet
type DocIndex = Map Word DocSet
\end{haskell}

The program spends most of its time building up the @DocIndex@ for the
whole set of documents, so that's where you want to focus your
attention.  There are two functions provided for building the index.
The first, @mkIndex@, builds a @DocIndex@ for a single document, given
its number and the document represented as a lazy @ByteString@ (if you
haven't come across the latter yet, don't worry - just think of it as
a string that is read from a file on demand).

\begin{haskell}
mkIndex :: Int -> L.ByteString -> DocIndex
\end{haskell}

The second function you need is @joinInidices@, which combines a list
of @DocIndex@ into a single @DocIndex@:

\begin{haskell}
joinIndices :: [DocIndex] -> DocIndex
\end{haskell}

The part of the program that builds the @DocIndex@ looks like this:

\begin{haskell}
      -- indices is a separate index for each (numbered) document
      indices :: [DocIndex]
      indices = zipWith mkIndex [0..] ss

      -- union the indices together to
      index = joinIndices indices
\end{haskell}

So there are two steps going on: first we call @mkIndex@ on all the
files @ss@, and then we combine all those @DocIndex@ values into a
single @DocIndex@.

Your task is to parallelise this as much as possible.  Don't expect to
get a \emph{lot} of speedup on this program: a factor of 2 on 4 cores
would be good.  The main goal is to get some speedup, and to get a
feel for what kinds of things are effective, while gaining some
familiarity with the programming models.

\begin{itemize}
\item You can use either the @Eval@ monad and Strategies, or the @Par@
monad, or even both if you like.  After trying with one, why not try
with the other.

\item Start by making the @mkIndex@ calls happen in parallel.  They
are completely independent, so this part should not be too hard.

\item The program builds a large data structure in the heap, and
spends a lot of time in the garbage collector.  So you might find that
it runs faster and scales better when given a larger "allocation
area":

   \begin{verbatim}
   $ echo "concurrent parallel" | ./index docs/* +RTS -s -A128M
   \end{verbatim}

\item Then look at @joinIndices@.  This is an associative operation:
for example, @joinIndices [a,b,c,d]@ is the same as

\begin{haskell}
  joinIndices [joinIndices [a,b],joinIndices [c,d]]
\end{haskell}

  So rather than a single @joinIndices@ call, we could express it as a
  tree-accumulation.  Your job is to experiment with different ways to
  build up the index using calls to @joinIndices@, and see how much
  parallel speedup you can get.

  Remember that rearranging the order of @joinIndices@ can have an
  effect on the sequential performance too - so if you improve over
  the performance of the original sequential program, use your program
  as the new baseline for calculating parallel speedup.
\end{itemize}

% Now try a larger example, a 16000-problem dataset:
% 
% {\small \begin{verbatim}
%   $ ./sudoku1 sudoku17.16000.txt +RTS -s
%   $ ./sudoku3 sudoku17.16000.txt +RTS -s -N2
% \end{verbatim}}
% 
% What speedup did you get?  Can you explain the discrepancy? (hint:
% look at the @SPARKS@ line in the @+RTS -s@ output.  GHC's spark pool
% can only hold about 8000 sparks, new sparks are discarded when the
% pool is full).
% 
% This example took a while to run, so let's simulate the problem on the
% smaller dataset by reducing the spark pool size to 500:
% 
% {\small \begin{verbatim}
%   $ ./sudoku3 sudoku17.1000.txt +RTS -s -N2 -e500
% \end{verbatim}}
% 
% Now, to avoid overflowing the spark pool we want to create fewer
% sparks, and so each spark needs to do more work.  Hence we need to
% divide the work into fewer, but larger, chunks.
% 
% @Control.Parallel.Strategies@ provides the following function:
% 
% {\small \begin{verbatim}
%      parListChunk :: Int -> Strategy a -> Strategy [a]
% \end{verbatim}}
% 
% The first argument to parListChunk is the chunk size in list elements,
% and the second is the Strategy to apply to the list elements.
% 
% \paragraph{Exercise 1.1.} Modify @sudoku3.hs@ to use @parListChunk@.
% 
% Try your version on the 1000-problem dataset with a spark pool size of 500:
% 
% {\small \begin{verbatim}
%   $ ./sudoku3 sudoku17.1000.txt +RTS -s -N2 -e500
% \end{verbatim}}
% 
% \noindent (if you want, try it on the 16000-problem dataset too)
% 
% What speedup did you get relative to sudoku1?  Try a few different
% chunk sizes.  Does it make any difference?
% 
% \paragraph{Exercise 1.2.} The Strategies library includes another
% operation for parallelising list-based operations:
% 
% {\small \begin{verbatim}
%     parBuffer :: Int -> Strategy a -> Strategy [a]
% \end{verbatim}}
% 
% Rather than dividing the list into chunks, parBuffer processes the
% list in a stream-like way, sparking N elements of the list ahead of
% the current element.  At any given point in time there should be no
% more than N active sparks.
% 
% Convert @sudoku3.hs@ to use @parBuffer@, and try it on the
% 1000-problem dataset.  What speedup did you get, and how does that
% compare to the results with @parListChunk@?
% 
% Try a few different buffer sizes - what difference do they make?


\newpage\Section{conc}{Lab 2: Concurrent Haskell}

The sample code corresponding to the examples in the notes can be
found in @par-tutorial/code@.  Try the @fork@ example from
Section~3.1:

{\small \begin{verbatim}
  $ ghc -threaded -rtsopts -eventlog --make fork.hs
  $ ./fork
  ABABABABABABABABABABABABABABABA...
\end{verbatim}}

\paragraph{Exercise 2.1.} In @par-tutorial/code/bingtranslator.hs@ there is
a program that translates a line of text into multiple languages using
the Bing Translate API.  For example:

{\small \begin{verbatim}
$ ./bingtranslator "translate this"
"translate this" appears to be in language "en"
ar: ترجمة هذا
bg: превод на това
zh-CHS: 翻译这
zh-CHT: 翻譯這
cs: přeložit
da: oversætte dette
... etc.
\end{verbatim}}

\noindent Compile the program - for this you may
need to install the @xml@ and @utf8-string@ packages first:

{\small \begin{verbatim}
  $ cabal install xml utf8-string
  $ ghc -threaded -rtsopts -eventlog --make bingtranslator.hs
\end{verbatim}}

When the program is compiled, run it yourself with some sample
text.\footnote{On Windows you may not see the international characters
  appear correctly on the console.  First, switch the codepage to
  UTF-8 with @chcp 65001@.  Then redirect the output of
  @bingtranslator@ to a file, and open the file in Notepad to see the
  output correctly.} Note how the translations appear slowly - the program queries the Bing API
for each translation in sequence.

\paragraph{Exercise 2.1.} Convert the program to perform all the translations
  concurrently.  Use the @Async@ API (the code is in the
  @bingtranslator.hs@ source file).

\paragraph{Exercise 2.2.} The program also makes two initial queries to
  the Bing API: one to get the list of supported languages, and
  another to detect the language in which the initial text is written.
  Make these two queries concurrently.

A sample solution can be found in @bingtranslatorconc.hs@.

\paragraph{Exercise 2.3.} Write a simple game that behaves as follows:
\begin{itemize}
\item A sequence of digits (initially empty) is displayed.
\item Each second, another digit is pushed on the left end of the sequence.
\item When the user types a digit, all matching digits are removed from the sequence.
\item If the sequence becomes ten digits long, the game is over.
\item The user scores one point for every digit successfully deleted.
\end{itemize}

You will need to start your program like this:

\begin{haskell}
main = do
 hSetBuffering stdout NoBuffering
 hSetBuffering stdin NoBuffering
 hSetEcho stdin False
\end{haskell}

\noindent to ensure that keypresses are received immediately and not
echoed to the screen.  You also need to be able to update the string
of digits on the screen; one way to do this is to move the cursor
backwards and overwrite the old string with spaces, and then write the
new string.  You can do this as follows:

\begin{haskell}
     putStr (replicate n '\8')
     putStr (replicate n ' ')
     putStr (replicate n '\8')
\end{haskell}

\noindent where @n@ is the length of the string you want to overwrite.

\textbf{Hints.}  One way is to structure the program as three threads,
with a single @MVar@.  The @MVar@ stores events, where an event is
either a key press, or a time event indicating that a second has elapsed and
another digit should be added.  One thread listens for key presses
(using @getChar@) and puts them in the @MVar@, another thread repeatedly
waits for one second (using @threadDelay@) and then puts the time
event in the @MVar@.  The third thread repeatedly takes an event from
the @MVar@ and updates the screen in response.

\paragraph{Exercise 2.3.1.} Make it so that new digits are pushed
faster as the current score increases.  For this you will need to
treat the current score as a piece of shared state between the timer
thread and the main thread.

\emph{Sample solution:} @code/game.hs@

\newpage\Section{stm}{Lab 3: Software Transactional Memory (STM)}

\paragraph{Exercise 3.1.} Implement a bounded channel type, with the
following signatures:

\begin{haskell}
data BoundedChan
newBoundedChan   :: Int -> STM (BoundedChan a)
readBoundedChan  :: BoundedChan a -> STM a
writeBoundedChan :: BoundedChan a -> a -> STM ()
\end{haskell}

\noindent @newBoundedChan@ takes the size of the channel, and
@writeBoundedChan@ blocks if the channel is full.

(After you've dong this, if you're feeling very brave, try implementing it
with @MVar@ instead of @STM@.  Obviously the functions must all be in
the @IO@ monad rather than @STM@.)

\paragraph{Exercise 3.2.} (optional: for the performance-obsessed)
Write a program in which two threads communicate over a channel, one
writing a large number of items to the channel and the other reading
them.  Measure the time it takes, using (a) the @Chan@ type, (b) the
@TChan@ type, (c) the bounded channel from Exercise 3.1 (choosing a
suitable bound).  Can you explain the differences in performance?

\paragraph{Exercise 3.3.} Write a program that corrects your
punctuation as you type.

In this exercise we are going to write a program that does the following:

\begin{itemize}
\item Accepts characters typed by the user, and echos them to the screen.

\item Automatically applies a correction to the input string.  The
correction is required to happen in a separate thread, because in
principle computing the correction might take time, and we want the
input to remain responsive.
\end{itemize}

Use the following guidelines to structure the program:

\begin{itemize}
\item Store the current input string in a @TVar@.
\item Use three threads:
      \begin{itemize}
      \item The main thread reads characters from @stdin@ and appends
      them to the string in the @TVar@.

      \item The @render@ thread watches for changes to the @TVar@ (as
      in the windowing example in the lecture/notes).  When a change
      is detected, it renders the new string.  Hint: you will need to
      delete the old string by emitting the correct number of @'\8'@
      backspace characters before printing the new string using
      @putStr@.

      \item The @correcter@ thread also watches for changes to the
      @TVar@, and when it detects a new string it applies a correction
      function to it and writes the result back to the @TVar@.  The
      correction function can do whatever you like; e.g. my correcter
      in the sample solution capitalises the first character after a
      full stop.
      \end{itemize}
\end{itemize}

Note that you will need to set @stdin@ and @stdout@ to no-buffering
mode, and disable echoing.  Your @main@ function should start like
this:

\begin{haskell}
import System.IO
import Control.Concurrent
import Control.Concurrent.STM

main = do
  hSetBuffering stdin  NoBuffering
  hSetBuffering stdout NoBuffering
  hSetEcho stdin False
  tvar <- newTVarIO ""
  forkIO (render tvar)
  forkIO (correcter tvar)
  ...
\end{haskell}

\noindent then you need to implement @render@, @correcter@, and the
code at the end of @main@ that reads the input characters.

\emph{Sample solution:} @code/correcter.hs@

\paragraph{Exercise 3.4.} Modify exercise 3.3 so that your text is
automatically translated into Japanese as you type it, using the Bing
translation service (see exercise 2.1) in the background.

\newpage\Section{servers}{Lab 4: Server applications}

In this exercise we'll add some extensions to the sample chat server.

Compile the chat server like so:

{\small \begin{verbatim}
$ cd code/chat
$ ghc --make -i.. Main.hs -o chat
[1 of 2] Compiling ConcurrentUtils  ( ../ConcurrentUtils.hs, ../ConcurrentUtils.o )
[2 of 2] Compiling Main             ( Main.hs, Main.o )
Linking chat ...
\end{verbatim}}

Check that you can run it and that it works properly.

{\small \begin{verbatim}
$ ./chat
Listening on port 44444
\end{verbatim}}

\noindent Now switch to another window, and connect a client:

{\small \begin{verbatim}
$ nc localhost 44444
What is your name?
a
*** a has connected
\end{verbatim}}

\noindent Now switch to a third window, and connect another client:

{\small \begin{verbatim}
$ nc localhost 44444
What is your name?
b
*** b has connected
\end{verbatim}}

\noindent when you connect client @b@, you should see a message on
client @a@'s session: @*** b has connected@.  Now try typing messages
into each client and check that the messages are broadcast properly.
Try kicking a client with @/kick b@.

The following exercises (4.1.1--4.1.6) are to add various enhancements
to the chat server, and they get progressively harder.  Feel free to
skip to exercise 4.2 at any time you like.

\paragraph{Exercise 4.1.1.} Add a @/names@ command to list the currently connected users.

\paragraph{Exercise 4.1.2.} Add a timeout to the "What is your name?"
question.  You probably want to use @System.Timeout.timeout@.

\paragraph{Exercise 4.1.3.} Add a timeout to the client loop: an
inactive client should be autonatically disconnected after a fixed
time limit.

\paragraph{Exercise 4.1.4.} @broadcast@ is inefficient because it uses an
unbounded transaction (see the section on performance of STM in the
notes).  Change the server to use a single broadcast channel; you can
either use @TChan@ with @dupTChan@, or alternatively build your own.
Note that this will mean that @runClient@ will need to check the
broacast channel in addition to its @clientSendChan@ and the
@clientKicked@ variable.

\paragraph{Exercise 4.1.5.} Add a @/nick@ command to change the current
client's name.  Careful! The name is stored as an immutable value in
the @Client@ record, and it is the key of the @clients@ map.  Some
refactoring of the program will be needed to make the name modifiable.
Hint: give each client a unique number, and use this as the key in the
map.

\paragraph{Exercise 4.1.6.} Add flood prevention: prevent a client from
issuing more than a certain number of messages in a given time.

\paragraph{Exercise 4.2.} The sample program in @code/arithgame.hs@ is a
simple single-player arithmetic game.  The exercise is to make it into
a multiplayer game, that works as follows:

\begin{itemize}
\item All current players are shown a random arithmetic question, e.g. @6 * 19@.
\item The first player to type in the correct answer gets a point; all the other players are notified who answered correctly and another question is shown.
\item If nobody answers correctly within ten seconds, another question is shown.
\item After ten questions, the game is over and the scores are shown.
\item New players can connect any time, and are asked for their name when connecting (as in the chat server).
\end{itemize}

You might want to start with the chat server as a basis for this
program, since some of the functionality is similar: clients need to
connect and tell the server their name.

\newpage\Section{distrib}{Lab 5: Distributed programming}

A key-value store is a simple database that
supports only operations to store and retrieve values associated with
keys.  Key-value stores have become popular over recent years because
they offer scalability advantages over traditional relational
databases, in exchange for supporting fewer operations (in particular,
database joins).

This exercise is to use the @remote@ framework to implement a
\emph{distributed fault-tolerant key-value store} in several stages.
You probably won't get to the end in the time available, but I think
it's a fun exercise and I hope you enjoy it nonetheless!

The interface exposed to clients is the following:

\begin{haskell}
type Database
type Key   = String
type Value = String

createDB :: ProcessM Database
set      :: Database -> Key -> Value -> ProcessM ()
get      :: Database -> Key -> ProcessM (Maybe Value)
\end{haskell}

\noindent where @createDB@ creates a database, and @set@ and @get@
perform operations on it.  The @set@ operation sets the given key to
the given value, and @get@ returns the current value associated with
the given key, or @Nothing@ if the key has no entry.

\paragraph{Exercise 5.1.} In @remote-db/db.hs@ I have supplied a
sample @main@ function that acts as a client for the database, and you
can use this to test your database.  The skeleton for the database
code itself is in @Database.hs@ in the same directory: the first
exercise is to implement a single-node database by modifying
@Database.hs@.  That is:

\begin{itemize}
\item @createDB@ should spawn a process to act as the database.  It
can spawn on the current node.
\item @get@ and @set@ should talk to the database process via
messages; you need to define the message type and the operations.
\end{itemize}

When you run @db.hs@, it will call @createDB@ to create a database,
and then populate it using the @Database.hs@ source file itself; every
word in the file is a key that maps to the word after it.  The client
will then lookup a couple of keys, and then go into an interactive
mode where you can type in keys that are looked up in the database.
Try it out with your database implementation, and satisfy yourself
that it is working.

\paragraph{Exercise 5.2.} The second stage is to make the database
\emph{distributed}. The basic plan is that we are going to divide up
the key space uniformly, and store each portion of the key space on a
separate node.  For example, operations on the key @"Simon"@ might go
to node 1, whereas operations on key @"Andres"@ go to node 2, and
@"Ralf"@ is handled by node 3.  The exact method you use for splitting
up the key space is up to you, but one simple scheme is to take the
first character of the key modulo the number of workers.

There will still be a single process handling requests from clients,
so we still have @type Database = ProcessId@.  However, this process
needs to delegate requests to the correct worker process according to
the key.

\begin{itemize}
\item Arrange to start worker processes on each of the nodes with
@WORKER@ role.  (for how to do this, see the example code for
multi-node ping in the notes, which is also in @remote-ping/ping-multi.hs@).

\item Write the code for the worker process.  You probably need to put
it in a different module (e.g. called @Worker@), due to restrictions
imposed by Template Haskell.  The worker process needs to maintain its
own @Map@, and handle get and set requests.

\item Make the main database process delegate operations to the
correct worker.  You should be able to make the worker reply directly
to the original client, rather than having to forward the response
from the worker back to the client.
\end{itemize}

Compile @db.hs@ against your distributed database, and make sure it
still works.

\emph{Sample solution:} @db2.hs@, @DatabaseDistrib.hs@, @Worker.hs@

\paragraph{Exercise 5.3.} Make the main database process monitor all the
worker processes.  Detect failure of a worker and emit a message using
@say@.  You will need to use @receiveWait@ to wait for multiple types
of message; see the @ping-fail.hs@ example for hints.

Note that we can't do anything sensible if a worker dies yet, that is
the next part of the exercise...

\paragraph{Exercise 5.4.} Implement \emph{fault tolerance} by
\emph{replication}.

\begin{itemize}
\item Instead of dividing the key space evenly across workers, put the
workers in pairs and give each pair a slice of the key space.  Both
workers in the pair will have exactly the same data.

\item Forward requests to both workers in the pair (it doesn't matter
that there will be two responses in the case of a @get@).

\item If a worker dies, you will need to remove the worker from your
internal list of workers so that you don't try to send it messages in
the future.\footnote{A real fault-tolerant database would restart the worker
on a new node and copy the database slice from its partner - by all
means have a go at doing this but the provided solutions don't do this.}
\end{itemize}

This \emph{should} result in a distributed key-value store that is
robust to individual nodes going down, at least if we don't kill too
many nodes too close together.

Try it out - kill a node while the database is running, and check that
you can still look up keys.  If you got this far, well done!

\emph{Sample solution:} @db4.hs@, @DatabaseRepl.hs@, @Worker.hs@

\newpage\Section{gpu}{Lab 6: GPU programming}

% working versions of packages:
% accelerate-cuda: f8c6d5e12673ac00eb88e58c3d5bdfb6d44bb708
% accelerate: 19c9df6fc91d8b7be6962538c80a77cc42fbc7ea

Your goal in this exercise is to \textbf{crack my password}.

My password is a dictionary word that can be found in the file

{\small \begin{verbatim}
/usr/share/dict/american-english
\end{verbatim}}

I have hashed the password using a checksum algorithm called CRC32
(32-bit Cyclic Redundancy Check), and the hash of my password is

\begin{center}
\textbf{\large 0xb4967c42}
\end{center}

In order to find the password, you will have to compute the CRC32
value for every word in the file @/usr/share/dict/american-english@, and then
find the word that has the CRC32 value @0xb4967c42@.

Some ordinary sequential Haskell code for computing CRC32 can be found
in @code/crc32/CRC32.hs@.  You can try it out by hand on a few words:

{\small \begin{verbatim}
> :load CRC32.hs
[1 of 1] Compiling CRC32            ( CRC32.hs, interpreted )
Ok, modules loaded: CRC32.
*CRC32> crc32String "hello"
3387906425
\end{verbatim}}

You can see the value in hex using @printf@:

{\small \begin{verbatim}
*CRC32> import Text.Printf
*CRC32 Text.Printf> printf "%x\n" crc32String "hello"
*CRC32 Text.Printf> printf "%x\n" (crc32String "hello")
c9ef5979
\end{verbatim}}


The CRC32 code is quite simple:

{\small \begin{verbatim}
crc32 :: [Word8] -> Word32
crc32 msg = go 0xffffffff msg
  where go crc []     = crc
        go crc (w:ws) = go crc' ws
           where crc' = (crc `shiftR` 8) `xor`
                        (crc32_tab !! fI (fI crc `xor` w))
                 fI x = fromIntegral x
\end{verbatim}}

It's basically a loop over the bytes of the input string, performing a
few bitwise operations at each stage.  It uses a 256-entry lookup
table @crc32_tab :: [Word32]@, which can also be found in the file
@CRC32.hs@, you can just import this module to gain access to it.

Our aim is to compute the CRC32 in parallel on all the words in the
dictionary, using the GPU.  The rest of this section will lead you to
the solution in several stages and give lots of hints along the way;
if you want a challenge then stop reading now and try to solve the
problem on your own!

\paragraph{Exercise 6.1.} write the following function:

\begin{haskell}
crc32_one :: Acc (Vector Word32) -> Exp Word32 -> Exp Word8
          -> Exp Word32
\end{haskell}

\noindent This performs one iteration of the CRC32 calculation.  The
first parameter is the lookup table, the second parameter is the
current CRC value, the third parameter is the byte of the input
string, and the output should be the new CRC32 value.

There is one extra criterion: \emph{if the input byte is zero, then
  the output CRC must be the same as the input}.  This will enable us
to run @crc32_one@ in parallel on multiple strings even though the
strings vary in length: we'll just pad the shorter strings with zeros.

\paragraph{Exercise 6.2.} Using @crc32_one@, write the following
function to compute the CRC32 values for a list of strings:

\begin{haskell}
crcAll :: [String] -> Acc (Vector Word32)
\end{haskell}

We're going to do this by mapping @crc32_one@ over a \emph{slice} of
the input strings.  That is, first we map over all the first
characters, then over all the second characters, and so on until we
have reached the maximum string length.

Start by bringing the lookup table @crc32_tab@ into the @Acc@ world:

\begin{haskell}
    table :: Acc (Vector Word32)
    table = ...
\end{haskell}

Next, you should calculate both the number of strings (call this @n@),
and their maximum length (call this @width@).

\begin{haskell}
    n = ...
    width = ...
\end{haskell}

Then create a vector of length @n@ containing the initial CRC values,
which are all @0xffffffff@.  Hint: use @fill@.

\begin{haskell}
    init_crcs :: Acc (Vector Word32)
    init_crcs = ...
\end{haskell}

Write a function @one_iter@, which performs one iteration over a
vector of input characters.  In here you will call @crc32_one@:

\begin{haskell}
    one_iter :: Acc (Vector Word32) -> Acc (Vector Word8)
             -> Acc (Vector Word32)
\end{haskell}

Finally, we want to write a function to perform all the iterations:

\begin{haskell}
    all :: Int -> Acc (Array Word32) -> [String]
        -> Acc (Array Word32)
    all 0 crcs words = crcs
    all x crcs words = ...
\end{haskell}

The first argument is a counter, which starts at @width@ and counts
down to zero.  The second argument is the vector of current CRC
values, and the third argument is the current list of strings, where
the first character of each is the next to process (at each iteration
we will remove one character from the head of each string).

In order to complete this function, you will need to create an
@Acc (Vector Word8)@ consisting of the first character from each
string (or zero if the string is already empty).  Then pass this to
@one_iter@ to calculate the new CRC values, and finally recursively
call @all@ with the new CRC values and the remainder of each string.

Test this out by calling it from GHCi with a few sample strings, and
test that you get the same results as calling the pure Haskell version
@crc32String@.

\paragraph{Exercise 6.3.} Find the index of the element in the array
that has the correct CRC32 value.

We want to write this function:

\begin{haskell}
find :: Acc (Vector Word32) -> Acc (Scalar Int)
\end{haskell}

\noindent which takes the array produced by @crcAll@, and returns the
index of the element that has the value we are looking for.

You could do this in two stages: first map every element to either (a)
its index if it has the correct value or (b) zero otherwise, and then
fold the @max@ function over the resulting array.  NB. there appears
to be a bug in Accelerate such that @max@ doesn't work on the GPU with
@Int@ arguments, so we have to use @Int32@.

Hint: an array of the same shape as an input array @arr@, in which
every element contains its index as an @Int32@ is given by:

\begin{haskell}
    generate (shape arr) (A.fromIntegral . unindex1)
\end{haskell}

\paragraph{Exercise 6.4.} Having done all this, you should be able to
use this @main@ wrapper to find the answer:

\begin{haskell}
main = do
  s <- readFile "/usr/share/dict/american-english"
  let ls = lines s
  let [r] = toList $ run $ find $ crcAll ls
  print (ls !! r)
\end{haskell}

You can run it with the interpreter, and it will take a few seconds.
To actually run it on the GPU, make sure you replace

\begin{haskell}
import Data.Array.Accelerate.Interpreter
\end{haskell}

\noindent with

\begin{haskell}
import Data.Array.Accelerate.CUDA
\end{haskell}

\noindent at the top of your program. Does it go faster on the GPU?

\emph{Sample solution:} @code/crc32/crc32_acc.hs@

%
% \subsection{Experiments with K-Means}
% 
% The code for K-Means is in @par-tutorial/code/kmeans@.  Compile it and
% run it to make sure you can get some parallel speedup.
% 
% Run the program like this:
% 
% {\small \begin{verbatim}
%  $ ./kmeans seq
% \end{verbatim}}
% 
% or
% 
% {\small \begin{verbatim}
%  $ ./kmeans par 1000 +RTS -N2
% \end{verbatim}}
% 
% to run the parallel version with 1000 chunks on 2 processors.
% 
% Note!  This program has to read the data set from disk each time it is
% run, which takes a substantial amount of time.  Therefore it includes
% code internally to measure the time taken to run the actual algorithm,
% and it prints the result at the end.  When computing parallel speedup,
% use this time, not the output from +RTS -s.
% 
% As written, the program is structured as map/reduce with a single
% level.  Instead we could structure it as divide and conquer to any
% depth, using the provided functions
% 
% \begin{haskell}
%   step   :: Int -> [Cluster] -> [Vector] -> [Cluster]
%   reduce :: Int -> [[Cluster]] -> [Cluster]
% \end{haskell}
% 
% \noindent \textbf{Exercise.} Restructure the program to use binary divide-and-conquer.
% 
% Hints: the input data set is a list of points and the algorithm
% needs to run repeatedly over it, so it will help to build the tree
% over which we run the divide-and-conquer algorithm up front.
% 
% Use a simple binary tree data structure:
% 
% \begin{haskell}
% data Tree a = Leaf a
%             | Node (Tree a) (Tree a)
% \end{haskell}
% 
% Write a function to convert the list of points into a tree, for
% example:
% 
% \begin{haskell}
% mkPointTree
%    :: Int            -- the depth at which to stop dividing
%    -> [Vector]       -- the set of points
%    -> Int            -- number of points
%    -> Tree [Vector]
% \end{haskell}
% 
% Next, write a function to implement the divide and conquer:
% 
% \begin{haskell}
% divconq :: Tree [Vector] -> [Cluster]
% \end{haskell}
% 
% It will help to make this a local function inside the loop function in
% @kmeans_par@, because it needs access to @clusters@, the current
% clusters, and @nclusters@.  At each node of the tree, compute the two
% branches in parallel.  Use either Strategies or the Eval monad
% directly.
% 
% Call divconq each time around the loop to compute the new set of
% clusters.
% 
% For the depth, try 10 (this will give $2^10$, i.e. 1024 leaves).
% 
% Run the program - how much speedup do you see?  How does it compare
% with the original one-level map/reduce?
% 
% Sample answer: @par-tutorial/kmeans/kmeans2.hs@
% 
% Bonus: modify the divide-and-conquer version to use the Par monad,
% 
% Hints:
% \begin{itemize}
% \item replace @import Control.Parallel.Strategies@ with
%     @import Control.Monad.Par@
% 
% \item the @divconq@ function needs to change:
% 
%   \begin{haskell}
%       divconq :: Tree [Vector] -> Par [Cluster]
%   \end{haskell}
% 
%     \noindent and use runPar when calling divconq.  For defining @divconq@ you
%     can use these from the @Control.Monad.Par@ library:
% 
%   \begin{haskell}
%       spawn :: a -> Par (IVar a)
%       get   :: IVar a -> Par a
%   \end{haskell}
% \end{itemize}
% 
% Sample answer: @par-tutorial/kmeans/kmeans3.hs@
% 
% \textbf{Exercise 4.} Implement the @MVar@ type using STM, in the most
% obvious way:
% 
% \begin{haskell}
% newtype TMVar a = TMVar (TVar (Maybe a))
% newEmptyTMVar :: STM (TMVar a)
% takeTMVar     :: TMVar a -> STM a
% putTMVar      :: TMVar a -> a -> STM ()
% \end{haskell}
% 
% This @MVar@ variant does not have the properties of \emph{fairness}
% and {single-wakeup} that the standard @MVar@ type does.  Can you
% modify the STM version so that it provides these properties?
% 
% \textbf{Exercise 5.} The operation @wait@ in the @Async@ API is not
% async-exception-safe.  Can you make it safe?
% 
% %      - make a channel implementation using Data.Seq.  Is it faster?
% %      - modify the server to only allow a fixed number of clients
% %      - write a program to generate load for the server. How many
% %        concurrent connections can it cope with?
% %      - geturlscancel: wait isn't async-safe, make it so

\end{document}
